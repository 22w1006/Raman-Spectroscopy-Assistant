% 为使用bibliography参考文献,编译选项选xelatex-bibtex-xelatex*,但很慢
% 不是最终版本,可以只选xelatex

% 本模板由松鼠制作
% github地址:github.com/Sciurus365/PhysChemLab
% 2018年1月23日

%推荐在编辑器窗口使用consolas+微软雅黑混合字体码代码==


\RequirePackage[l2tabu, orthodox]{nag}
\documentclass[UTF8]{article}


\usepackage{threeparttable}
\usepackage[zihao=-4]{ctex}
\usepackage{graphicx}
\usepackage{multicol}
\usepackage[a4paper]{geometry}
\usepackage{booktabs}
\usepackage{microtype}
\usepackage{siunitx}
\usepackage{ragged2e}
\usepackage{fontspec}
\usepackage{multirow}
\usepackage{mhchem}
\usepackage[square,sort,comma,numbers,super]{natbib}
\usepackage{fancyhdr}
\usepackage{url}

\usepackage{amsmath}%输入特殊数学符号
\usepackage{float}%浮动体控制
\usepackage[figuresright]{rotating}%竖排表格


% 纸张、页边距设置
\graphicspath{{figures/}}
\geometry{left=2.0cm,right=2.0cm,top=2.0cm,bottom=2.0cm}

% 统一全文英语字体
\usepackage{unicode-math}
\setmathfont{XITS Math}
%\setmainfont{Times New Roman}
\setmainfont{XITS Math}


% 实验相关信息设置
\newcommand{\expname}{中级物理化学实验报告}
\newcommand{\expdate}{日期}
\newcommand{\exptemperature}{0}
\newcommand{\exppressure}{0}
\newcommand{\expteacher}{指导教师}



% 页眉页脚设置
\pagestyle{fancy}  
\lhead{拉曼光谱仪搭建与应用}  
\chead{\expname}  
\rhead{\expdate}  
\lfoot{}  
\cfoot{\thepage}  
\rfoot{}  
\renewcommand{\headrulewidth}{0.4pt}  
\renewcommand{\footrulewidth}{0.4pt}  

% 温度 电动势 浓度和气压的快速输入
\newcommand{\swd}[1]{\SI{#1}{\degreeCelsius}}
\newcommand{\sdds}[1]{\SI{#1}{\volt}}
\newcommand{\snd}[1]{\SI{#1}{\mole \per \liter}}
\newcommand{\sqy}[1]{\SI{#1}{\kilo \pascal}}
% 摄氏度、公式中文本、\varepsilon的快速输入
% \newcommand{\dC}{\si{\degreeCelsius}}
\newcommand{\tr}[1]{\textrm{#1}}
\newcommand{\ve}{\varepsilon}

\def\dC{\si{\degreeCelsius}}
\def\kPa{\,\si{kPa}}
\newcommand{\dw}[1]{\,\mathrm{#1}}

\def\d{\mathrm{d}}
\def\e{\mathrm{e}}
\def\i{\mathrm{i}}
\def\p{\partial}
\def\toinfty{\to \infty}


% 如果需要在三线表中插入竖线,请进行以下设置以避免竖线被割断
%\belowrulesep=0ex
%\aboverulesep=0ex

% 使用siunitx包报告不确定度时,以下设置可以使结果以\bar(X) \pm \sigma 的形式表示
% 使用siunitx包书写单位时,以下设置可以使单位之间加\cdot点
\sisetup{
	separate-uncertainty = true,
	inter-unit-product = \ensuremath{{}\cdot{}}
}

\begin{document}
	
	%——————————封面页——————————
	\begin{titlepage}
		\vspace*{1cm}
		\begin{figure}[h]
			\centering
			\includegraphics[width=0.7\linewidth]{logo}
		\end{figure}
		
		\vspace*{0.5cm}
		
		\begin{center}
			\Huge{\textbf{拉曼光谱仪搭建与应用}}
			
			\Large{\expname}
		\end{center}
		
		\vspace*{0.5cm}
		
		\begin{table}[h]
			\centering
				\begin{tabular}{cccc}
					元培学院 & 肖舒凡 \\
					化学与分子工程学院  & 曾志炜 \\
					化学与分子工程学院  & 傅洪鑫 \\
				\end{tabular}
		\end{table}
	
	\vspace*{1cm}
	
	\textbf{摘要}\quad  摘要
	
	\end{titlepage}
	
	\normalsize

	\section{引言}
	拉曼光谱是一种分析材料结构和化学成分的有效手段。拉曼光谱基于光的非弹性散射,广泛应用于材料科学、生物医学、环境监测等领域。本实验的目标是搭建自制的拉曼光谱仪,并探索其原理和应用。
	

	\section{基本原理}
	\subsection{拉曼散射原理}
	这里写拉曼散射的原理
	
	\subsection{拉曼光谱仪器构造}
	要配理论图

	\subsection{表面增强拉曼光谱原理}
	这里简单介绍表面增强的原理


	\section{软件开发}
	用LabView开发软件,实现xx功能,略写即可。
	
	详见github项目 https://github.com/22w1006/Raman-Spectroscopy-Assistant。(加个链接)
	
	
	\section{实验步骤}
	\subsection{实验用品}
	\textbf{元件} \quad 532 nm 激光器
	
	\textbf{试剂} \quad 乙醇、丙酮等
	
	\textbf{仪器} \quad 电磁搅拌器、离心机、超声机等

	\subsection{拉曼光谱仪的搭建、标定与测试}
	根据原理图搭建拉曼光谱。配实际仪器图

	标定过程,文字描述。

	测试过程,文字描述。

	\subsection{纳米银溶胶的制备}

	\subsection{样品测试}
	\subsubsection{测试1:己胺、苯胺、对硝基苯胺}
	\subsubsection{测试2:表面增强}
	\subsubsection{测试3:水和重水}

	
	\section{实验结果与数据分析}
	\subsection{拉曼光谱的基本结果}
	光谱结果:乙醇、丙酮、己胺、苯胺、对硝基苯胺、水、重水、硫酸钠。

	光谱性能表征:光谱范围、分辨率、信噪比。

	\subsection{纳米银溶胶的表面增强拉曼光谱结果展示与分析}
	增强前后对比,增强数量级说明。

	银溶胶表征。

	\subsection{水和重水混合物电解前后的拉曼光谱结果展示与分析}
	电解前后峰值对比,相对含量变化说明。

	
	\section{结果讨论与解释}
	\subsection{己胺、苯胺、对硝基苯胺的Gaussian计算(可选,待定)}
	

	\subsection{纳米银溶胶的表面增强拉曼光谱效应解释}
	讨论一下表面增强的实验结果,和上面的结果展示看着各自的内容来写就行。比如如何确定有增强效果,对不同物质的增强效果。

	\subsection{水和重水混合物电解过程解释 }
	电解水实验相关计算、解释和讨论
	
	\section{结论}
	\subsection{总结实验}
	xx,达成既定目标。

	\subsection{对实验的改进和未来研究方向的展望}
	仪器进一步优化;软件的进一步开发:CCD噪点去除;表面增强原理探索
	
	
	\bibliographystyle{ieeetr}
	\bibliography{reference}
	% 原生bibtex对中文支持不好,请在bibtex文献库对中文文献做以下调整:
	% 将作者以 “张三{, }李四{, }松鼠” 形式表示
	% 将版本号输入在书名一栏 例如 “物理化学实验(第四版)”


	
\end{document}
